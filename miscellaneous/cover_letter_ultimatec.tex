%! Author = anton
%! Date = 9/10/24

\documentclass[9pt]{article}

\usepackage{amsmath, amssymb, amsfonts}
\usepackage{hyperref}
\usepackage[english, russian]{babel}

% Document
\begin{document}

    Здравствуйте, команда Ультиматек!

    Меня заинтересовала вакансия разработчика математических моделей
    так как уже имею и хочу далее развивать знания в математике и навыки в написании кода.

    Задачами оптимизации начал заниматься с началом работы над системой
    \href{https://en.wikipedia.org/wiki/Simultaneous_localization_and_mapping}{SLAM}.
    Первым стоящим упоминания результатом было выравнивание
    (регистрация) 2х изображений на группе $\mathbb{SL}(3)$ методом Гаусса-Ньютона.
    Сам код был написан с нуля на c++ для CUDA.

    Вторым интересным результатом могу назвать выравнивание SLAM и GNSS траекторий (слайды | видео).
    Первая может накапливать одометрическую ошибку, вторая может иметь выбросы в виде кусочно постоянного сдвига.
    Задачей было выравнять (регистрация с деформацией) SLAM траекторию к GNSS траектории,
    нивелируя ошибки первой и игнорируя выбросы второй.
    Здесь использовал оптимизацию ADM, в качестве sub solver'ов -- google ceres и ECOS.

    Для общего развития изучал теорию вариационного исчисления и оптимизации на графах (для задачи нахождения соответствий в
    стереопаре), практических результатов в этих направлениях нет (моё изложении теории для графах).
    Из готовых solver'ов (кроме упомянутых выше) преимущественно использовал BFGS (из пакета scikit-learn) и ADAM (для обучения нейросетей).

    Кроме оптимизации, изучал и применял на практике аппарат теории вероятностей и спектральную теорию в
    (оснащённых) Гильбертовых пространствах.
    

    Мои контакты: \\

    \begin{tabular}{ll}
            phone: & {\tt +7 918 513 80 20} \\
            telegram: & {\tt the$\_$m0rzh} \\
            e-mail:& {\tt morzhakovanton@gmail.com} \\\label{tab:table}
    \end{tabular}


    С уважением,
    Антон Моржаков.

\end{document}

Твои задачи:
    Разработка и реализация математических моделей для задач диагностики, в том числе в техническом обслуживании, качестве продукции и прочих прикладных областях.
    Сбор и анализ данных, необходимых для построения и валидации моделей.
    Применение статистических методов и алгоритмов машинного обучения для обработки и интерпретации данных.
    Разработка алгоритмов для реализации моделей в программном обеспечении.
    Сотрудничество с экспертами в области диагностики для уточнения требований и функциональных возможностей моделей.
    Подготовка технической документации и отчетов по результатам исследований.
    Работа в мультидисциплинарной команде для интеграции математических моделей в комплексные диагностические решения.
    Постоянное обучение и ознакомление с современными методами математического моделирования и анализа данных.

Мы ждем от тебя:
    Высшее образование в области математики, прикладной математики, физики, информатики или смежных дисциплин.
    Опыт работы с математическим моделированием, статистическим анализом данных и машинным обучением от 4 лет
    Знание программных средств для обработки данных и моделирования (например, R, Python, MATLAB).
    Умение работать с большими данными и открытыми базами данных.
    Владение методами оптимизации и случайных процессов.
    Способность аналитически мыслить и решать сложные задачи.
    Уверенное владение английским языком, как устным, так и письменным, для чтения специализированной литературы и общения с международными партнерами.
    Желательно наличие публикаций или докладов в области математического моделирования или диагностики.